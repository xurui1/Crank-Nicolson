\documentclass[11pt]{article} % use larger type; default would be 10pt
\usepackage{mathrsfs,amsmath}
\usepackage{gensymb}
\usepackage{graphicx}
\usepackage{float}
\graphicspath{ {images/} }
\usepackage{fixltx2e} %for subscript
\usepackage[margin=0.8in]{geometry}
\setlength{\parskip}{\baselineskip}%
\setlength{\voffset}{0in}
%\pagenumbering{gobble}

\title{Solving the modified diffusion equation using the Crank-Nicolson Algorithm}
\author{Rui Xu}
%\date{} % Activate to display a given date or no date (if empty),
         % otherwise the current date is printed 
         
\begin{document}

\maketitle

\section*{The modified Diffusion Equation}
 
 We wish to solve a modified diffusion equation in Cartesian, cylindrical, and spherical coordinates using the Crank-Nicolson algorithm. We assume y and z symmetry in Cartesian coordinates, azimuthal and vertical symmetry in cylindrical coordinates, and azimuthal and polar symmetry in spherical coordinates. This makes for a one-dimensional computational box in the `$\textbf{x}$' direction for the Cartesian coordinate system, and one-dimensional computational boxes in the radial `$\textbf{r}$' direction for the cylindrical and  spherical coordinate systems. For the purpose of this derivation, $\textbf{x}=\textbf{r}$, and we use $\textbf{r}$ as our generalized coordinate.
 
We are solving the chain propagator for the continuous Gaussian chain representation of a polymer, where $q(\textbf{r}, s)$ is an end-integrated chain propagator and `$s$' is an index that runs along the length of the chain. The modified diffusion equation has the form: 
 \begin{equation}
\frac{\partial q(\textbf{r}; s)}{\partial s} = C \nabla^2 q(\textbf{r}; s) - \omega (\textbf{r}) q(\textbf{r}; s)
 \end{equation}
\noindent
where $C$ is the diffusion coefficient that represents $R_g^2$, the radius of gyration of the polymer squared, or $b^2/6$, where $b$ is the persistence length of the polymer. The $\omega$-field represents an auxiliary field coupled to polymer density. 

We solve the modified diffusion equation in three coordinate systems, each with a different Laplacian. The Laplacian in Cartesian coordinates, with y and z symmetry and replacing $\textbf{x}$ with $\textbf{r}$ is:
\begin{equation}
\nabla^2 = \frac{\partial^2}{\partial r^2} 
\end{equation}
\noindent
The Laplacian in cylindrical coordinates, assuming azimuthal and vertical symmetry, has the form:
\begin{equation}
\nabla^2=\frac{1}{r} \frac{\partial}{\partial r} \left(r \frac{\partial}{\partial r} \right) =\frac{\partial^2}{\partial r^2} + \frac{1}{r} \frac{\partial}{\partial r}
\end{equation}
\noindent 
The Laplacian in spherical coordinates, assuming azimuthal and polar symmetry, has the form:
\begin{equation}
\nabla^2=\frac{1}{r^2} \frac{\partial}{\partial r} \left(r^2 \frac{\partial}{\partial r} \right) =\frac{\partial^2}{\partial r^2} + \frac{2}{r} \frac{\partial}{\partial r}
\end{equation}
\noindent
Instead of deriving the Crank-Nicolson algorithm separately for each coordinate system, we use a generalized Laplacian:
\begin{equation}
\nabla^2=\frac{\partial^2}{\partial r^2} + \frac{\left(\textbf{D}-1\right)}{r}\frac{\partial}{\partial r}
\end{equation}
\noindent
where $\textbf{D}=1$ for Cartesian coordinates, $\textbf{D}=2$ for cylindrical coordinates, and $\textbf{D}=3$ for spherical coordinates. Inserting Equation 5 into the modified diffusion equation gives:
\begin{equation}
\frac{\partial q(\textbf{r}; s)}{\partial s} = C \left(\frac{\partial^2}{\partial r^2} + \frac{(\textbf{D}-1)}{r} \frac{\partial}{\partial r}    \right) q(\textbf{r}; s) - \omega (\textbf{r}) q(\textbf{r}; s)
\end{equation}\\[12pt]


We define a computational grid of discrete points at which the continuous function $q(\textbf{r};s)$ is sampled. We define the computational grid as having length $L_r$ where $\textbf{r} \in [0,L_r]$. We use $N_r$ equally spaced points:
\begin{equation}
r_i = i \Delta r, \quad i = 0 ... N_r - 1
\end{equation}
\noindent
for $\Delta r = L_r/(N_r-1)$, the chosen grid spacing in the r-direction. We also discretize the continous Gaussian chain over the interval $s \in [0,N]$ using $N_s$ equally spaced points:
\begin{equation}
s_n = n \Delta s, \quad n = 0 ... N_s -1
\end{equation}
\noindent
for $\Delta s = N/(N_s-1)$, the contour step along $q(r,z;s)$. We now change our notation for the end-integrated chain propagator from its continuous form, $q(\textbf{r};s)$, to its discretized form, $q_{i}^n$ which we will use for the remainder of the derivation. As an initial condition, we set:
\begin{equation}
q_{i}^0 = 1 \quad \text{for} \quad i \in [0,N_r-1]
\end{equation}

The Crank-Nicolson algorithm is an implicit method that consists of forward Euler difference approximation in the $\textbf{s}$ domain, and a combination of the forward Euler method at the $n^{\text{th}}$ monomer, and the backward Euler method at the $n+1$ monomer. In the following section, we implement the Crank-Nicolson algorithm for our modified diffusion equation. 

\section*{Crank-Nicolson}
The next step is to implement a finite differencing method to the modified diffusion equation.  For the $s$-derivative, we use a forward Euler difference approximation:
\begin{equation}
\frac{\partial q_i^n}{\partial s} \Rightarrow \dfrac{q_{i}^{n+1} - q_{i}^n}{\Delta s}
\end{equation}

\noindent
For the first order $r$-derivative, we use the following difference approximation:
\begin{equation}
\frac{\partial q_i^n}{\partial r}  \Rightarrow \frac{1}{2} \left[\left(\frac{q_{i+1}^{n+1} - q_{i-1}^{n+1}}{2(\Delta r)} \right)+ \left(\frac{q_{i+1}^{n} - q_{i-1}^{n}}{2(\Delta r)} \right) \right]
\end{equation}

\noindent
For the second order $r$-derivative, we use the following central difference approximation:
\begin{equation}
\frac{\partial^2 q_i^n}{\partial r^2} \Rightarrow \frac{1}{2(\Delta r)^2} \textbf{\Big[} \left( q_{i+1}^{n+1} -2q_{i}^{n+1} + q_{i-1}^{n+1} \right)+\left( q_{i+1}^{n} -2q_{i}^{n} + q_{i-1}^{n} \right) \textbf{\Big]}
\end{equation}

\noindent
After inserting the difference approximations into the modified diffusion equation, the modified diffusion equation has the form:
\begin{align}
\frac{1}{\Delta s} \left(q_{i}^{n+1} - q_{i}^n \right) &=  \frac{C}{2(\Delta r)^2} \textbf{\Big[} \left( q_{i+1}^{n+1} -2q_{i}^{n+1} + q_{i-1}^{n+1} \right)+\left( q_{i+1}^{n} -2q_{i}^{n} + q_{i-1}^{n} \right) \textbf{\Big]} \\ 
&+  \frac{C (\textbf{D}-1)}{2r} \left[\left(\frac{q_{i+1}^{n+1} - q_{i-1}^{n+1}}{2(\Delta r)} \right)+ \left(\frac{q_{i+1}^{n} - q_{i-1}^{n}}{2(\Delta r)} \right) \right] \\
&-\frac{\omega_i}{2} \left(q_i^{n+1}+q_i^n \right)
\end{align}
\noindent
The next step is to separate the $n+1$ terms from the $n$ terms:
\begin{align}
&\frac{1}{\Delta s}q_{i}^{n+1}  - \frac{C}{2(\Delta r)^2} \left( q_{i+1}^{n+1} -2q_{i}^{n+1} + q_{i-1}^{n+1} \right) - \frac{C(\textbf{D}-1)}{4r\Delta r} \left( q_{i+1}^{n+1} - q_{i-1}^{n+1} \right)  + \frac{\omega_{i}}{2}  q_{i}^{n+1} \\
& = \frac{1}{\Delta s} q_{i}^n  + \frac{C}{2(\Delta r)^2} \left( q_{i+1}^{n} -2q_{i}^{n} + q_{i-1}^{n} \right) + \frac{C(\textbf{D}-1)}{4r\Delta r} \left( q_{i+1}^{n} - q_{i-1}^{n} \right) - \frac{\omega_{i}}{2} q_{i}^{n} 
\end{align}
\noindent
The next step is to group the coefficients by their coordinates ($i+1,i,i-1$). We multiply the entire expression by $\Delta s$, set C to 1, and use functions $\alpha_{+1,0,-1}$ and $\beta_{+1,0,-1}$ to simplify the expression:
\begin{equation}
\alpha_{+1} q_{i+1}^{n+1} + \alpha_0 q_{i}^{n+1} + \alpha_{-1} q_{i-1}^{n+1} = \beta_{+1} q_{i+1}^{n+1} + \beta_0 q_{i,j}^{n+1} + \beta_{-1} q_{i-1}^{n+1}
\end{equation}

\noindent
where:
\begin{align}
\alpha_{+1} &\equiv - \dfrac{ (\Delta s)}{2(\Delta r)^2} - \dfrac{(\textbf{D-1}) (\Delta s)}{4r(\Delta r)} \\
\alpha_0 &\equiv 1 + \dfrac{(\Delta s)}{(\Delta r)^2} + \dfrac{(\Delta s)}{2} \omega_{i} \\
\alpha_{-1}  &\equiv -  \dfrac{ (\Delta s)}{2(\Delta r)^2} + \dfrac{(\textbf{D-1}) (\Delta s)}{4r(\Delta r)} \\
\beta_{+1}  &\equiv  \dfrac{ (\Delta s)}{2(\Delta r)^2} + \dfrac{(\textbf{D-1}) (\Delta s)}{4r(\Delta r)} \\
\beta_0  &\equiv 1 - \dfrac{ (\Delta s)}{(\Delta z)^2} - \dfrac{(\Delta s)}{2} \omega_{i} \\
\beta_{-1}  &\equiv  \dfrac{ (\Delta s)}{2(\Delta r)^2} - \dfrac{(\textbf{D-1}) (\Delta s)}{4r(\Delta r)} 
\end{align} \\[12pt]

We implement a zero derivative boundary condition (Neumann boundary condition). The mathematical form of this boundary condition is the following:
\noindent
\begin{equation}
\dfrac{\partial q_{0}}{\partial r} = \dfrac{\partial q_{N_r-1}}{\partial r} = 0
\end{equation}
\noindent
This requires that $q_{1} = q_{-1}$, and $q_{N_r-2} = q_{N_r,j}$.  The matrix form of the modified diffusion equation is the following:


\[  \begin{bmatrix}
q_{0}^{n+1} \\[0.5em]
q_{1}^{n+1} \\[0.5em]
\vdots \\[0.5em]
q_{N_r-2}^{n+1} \\[0.5em]
q_{Nr-1}^{n+1} \\[0.5em]
\end{bmatrix} = 
%
\begin{bmatrix}
\alpha_0 & (\alpha_{+1}+\alpha_{-1}) & 0 & \cdots & 0 \\[0.5em]
\alpha_{-1} & \alpha_0 & \alpha_{+1} &\cdots &0 \\[0.5em]
\vdots & \ddots & \ddots & \ddots & \vdots \\[0.5em]
0 & \cdots & \alpha_{-1} & \alpha_0 & \alpha_1  \\[0.5em]
0 & \cdots & 0 & (\alpha_1+\alpha_{-1}) & \alpha_0 \\[0.5em]
\end{bmatrix}^{-1} 
%
\begin{bmatrix}
(\beta_{+1}+\beta_{-1})q_{1}^{n} + \beta_0q_{0}^{n} \\[0.5em]
\beta_{+1}q_2^n +\beta_0 q_{1}^{n} + \beta_{-1}q_0^n \\[0.5em]
\vdots \\[0.5em]
\beta_{+1}q_{Nr-1}^n +\beta_0 q_{Nr-2}^{n} + \beta_{-1}q_{Nr-3}^n \\[0.5em]
(\beta_{+1}+\beta_{-1})q_{Nr-2}^{n} + \beta_0q_{Nr-1}^{n} \\[0.5em]
\end{bmatrix}
\]

This matrix is tridiagonal, which means that we can solve this matrix algebra problem with the Tridiagonal Matrix Algorithm (TDMA). This algorithm scales as $O(N)$, significantly better than Gaussian elimination. We solve the diffusion equation by propagating the solution from $n=0$ to $n=N_s-1$.
 
\end{document}